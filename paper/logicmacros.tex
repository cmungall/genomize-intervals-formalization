%%%%%%%%%%%%%%%%%%%%%%%%%%%%%%%%%%%%%%%%%%%%%%%%%%%%%%%%%%%%%%%
%%                    ** mymacros.tex **                     %%
%%%%%%%%%%%%%%%%%%%%%%%%%%%%%%%%%%%%%%%%%%%%%%%%%%%%%%%%%%%%%%%


\typeout{** loading mymacros.tex **}

\def\h{\hbox}

\newcommand{\mathcom}[3]{ \newcommand{#1}[#2]{\mbox{$#3$}}}
\newcommand{\remathcom}[3]{ \renewcommand{#1}[#2]{\mbox{$#3$}}}

\newcommand{\mdef}[2]{ \newcommand{#1}{\mbox{$#2$}} }
 
\mathcom{\y}{0}{\ \vdash\ }               % yeilds 
\mathcom{\ys}{0}{\vdash_{\cal S}}         % yeilds sub S
\mathcom{\yt}{0}{\vdash_{\theta}}         % yeilds sub theta

\mathcom{\absurd}{0}{\mathbf{f}}                 % absurdity
\mathcom{\imp}{0}{\ \rightarrow\ }            % implication arrow
\mathcom{\rimp}{0}{\ \leftarrow\ }            % implication arrow

\mathcom{\con}{0}{\ \wedge\ }                 % conjunction
\mathcom{\dis}{0}{\ \vee\ }                   % disjunction
\mathcom{\n}{0}{\neg}                     % negation
\mathcom{\dimp}{0}{\ \leftrightarrow\ }       % mat equiv

\mathcom{\corresponds}{0}{\ \Lleftarrow\! \! \Rrightarrow\ }


%%\mathcom{\th}{0}{\theta}                  % theta


\mathcom{\A}{0}{\forall}                  % universal quantifier
\mathcom{\E}{0}{\exists}                  % existential quant

\mathcom{\Dec}{1}{{\cal D}(#1)}           % D(#1)

\mathcom{\elt}{0}{\in}

\mathcom{\tuple}{1}{\langle #1 \rangle}

% \mathcom{\equivdef}{0}{\equiv_{\mbox{\em \tiny def}}}
% \mathcom{\eqdef}{0}{=_{\mbox{\em \tiny def}}}
\def\equivdef{\mathrel{\ \equiv_{\mbox{\em \tiny
def}}\ }}

\def\eqdef{\mathrel{\ =_{\mbox{\em \tiny def}}\ }}


% \newtheorem{Rule}{Rule}
% \newtheorem{Lemma}{Lemma}
% \newtheorem{Theorem}{Theorem}


% Redeclaration of symbols for intuitionistic logic
% see Diller p.126
\def\ineg{\mathop\sim}
\def\iimp{\mathbin\Rightarrow}
\def\iiff{\mathbin\Leftrightarrow}
% new symbols from amssymb
%\def\iimp{\mathbin\rightarrowtail}
%\def\iiff{\mathbin\leftrightsquigarrow}

\def\icon{\mathbin\curlywedge}
\def\idis{\mathbin\curlyvee}

%Definitions for RCC inserts
\newcommand{\notch}{\Longrightarrow\kern-16pt{/}\ \ \ }
\newcommand{\ch}{\Longrightarrow}
\newcommand{\pr}[1]{\mbox{\sf #1}}

% RCC relations
\mathcom{\C}{0}{\pr{C}}
\remathcom{\P}{0}{\pr{P}}
\remathcom{\O}{0}{\pr{O}}
\mathcom{\PO}{0}{\pr{PO}}
\mathcom{\PP}{0}{\pr{PP}}
\mathcom{\NTPP}{0}{\pr{NTPP}}
\mathcom{\NTPPI}{0}{\pr{NTPP}^{-1}}
\mathcom{\TPP}{0}{\pr{TPP}}
\mathcom{\TPPI}{0}{\pr{TPP}^{-1}}
\mathcom{\TP}{0}{\pr{TP}}
\mathcom{\NTP}{0}{\pr{NTP}}
\mathcom{\NTPi}{0}{\pr{NTPi}}
\mathcom{\EC}{0}{\pr{EC}}
\mathcom{\DC}{0}{\pr{DC}}
\mathcom{\DR}{0}{\pr{DR}}
\mathcom{\EQ}{0}{\pr{EQ}}
\mathcom{\CG}{0}{\pr{CG}}

\mathcom{\Pin}{0}{\pr{Pi}} % \Pi is Greek letter
\mathcom{\PPi}{0}{\pr{PPi}}
\mathcom{\NTPPi}{0}{\pr{NTPPi}}

\mathcom{\OP}{0}{\pr{OP}}

\mathcom{\conv}{0}{\pr{conv}}
\mathcom{\rsum}{0}{\pr{sum}}
\mathcom{\rdiff}{0}{\pr{diff}}
\mathcom{\rcompl}{0}{\pr{compl}}
\mathcom{\rprod}{0}{\pr{prod}}
\mathcom{\cp}{0}{\pr{cp}}
\mathcom{\Us}{0}{\pr{Us}}
\mathcom{\us}{0}{\pr{u}}
\mathcom{\NULL}{0}{\pr{NULL}}
\mathcom{\rnull}{0}{\emptyset}
\mathcom{\CONV}{0}{\pr{CONV}}
\mathcom{\CON}{0}{\pr{CON}}

% Adaptations for new latex compatibility
\renewcommand{\Box}{\square}
\def\cal{\mathcal}

% meta language in definitions etc
\mathcom{\ml}{1}{\;\;\;\; \mbox{#1} \;\;\;\;}


%-- TOPLOG SYMBOLS

\mathcom{\Co}{0}{{\cal C}_{0} \ }
\mathcom{\CoX}{0}{{\cal C}_{0}}
\mathcom{\Cop}{0}{{\cal C}_{0}^+ \ }
\mathcom{\CopX}{0}{{\cal C}_{0}^+}
\mathcom{\mcop}{0}{\models_{C_o^+}}

\mathcom{\Lop}{0}{{\cal L}_{0}^+ \ }
\mathcom{\LopX}{0}{{\cal L}_{0}^+}
\def\mlo{\models_{{\mathcal  L}_0}}
\def\mlop{\models_{{\mathcal  L}_0^+}}

\mathcom{\Ci}{0}{{\cal C}_{1} \ }
\mathcom{\CiX}{0}{{\cal C}_{1}}

\mathcom{\Io}{0}{{\cal I}_{0} \ }
\mathcom{\IoX}{0}{{\cal I}_{0}}
\mathcom{\Iop}{0}{{\cal I}_{0}^+ \ }
\mathcom{\IopX}{0}{{\cal I}_{0}^+}

%%\mathcom{\inter}{1}{\langle #1 \rangle}
\mathcom{\inter}{1}{i( #1 )}

\newfont{\myssfont}{cmss12}

\mathcom{\Maps}{2}{\, _{#1}\!\!\!\rightleftharpoons^{#2} \ }

\mathcom{\CotoST}{0}{\Maps{C_0}{ST}}
\mathcom{\CtoST}{0}{\Maps{C\,}{ST}}
\mathcom{\IotoST}{0}{\Maps{I_0}{ST}}

\mathcom{\uni}{0}{\cal U \, }
\mathcom{\uniX}{0}{\cal U}
\mathcom{\U}{0}{\cal U}

\mathcom{\eu}{0}{\,=\,\uni}

\mathcom{\ol}{1}{\overline{#1}}

\mathcom{\Lsse}{0}{{\cal L}_{sse}}
\mathcom{\Lssei}{0}{{\cal L}_{ssei}}

\mathcom{\Luse}{0}{{\cal L}_{use}}
\mathcom{\Lusei}{0}{{\cal L}_{usei}}

\mathcom{\disdots}{0}{\dis\!\!\!\ldots\!\!\dis}
\mathcom{\condots}{0}{\con\ldots\con}





%% Miscelleneous


\def\HA#1#2{\pr{Holds-At}(#1,#2)}
\def\col{\!:\!}

% Enclose arguments in square double brackets to
% refer to semantic value of an expression
\def\semvalue#1{[\mkern-3mu[#1]\mkern-3mu]}

\def\forces{\mathop{\setbox0=\hbox{$\vdash$}\ \rule{0.04em}{1\ht0}\mkern1.5mu\box0}}

\mathcom{\hence}{0}{\Longrightarrow \hspace{2em}}

\def\sliderule{\centerline{\rule{4in}{1mm}}}

\def\Cbox{\mathop{\square\mkern-10mu
                   \mbox{\raise0.45ex\hbox{\tiny\sf C}}
                  \mkern2mu
                 }
         }

\def\letterbox#1{\mathop{\square\mkern-10mu
                   \mbox{\raise0.2ex\hbox{\small\sf #1}}
                  \mkern2mu
                 }
         }

%\def\tBox{\letterbox{\lower0.4ex\hbox{t}}}
\def\sBox{\letterbox{s}}
\def\tBox{\letterbox{t}}

\def\overstrike#1#2{\mathop{%
               \setbox1=\hbox{$#1$}%
               \setbox2=\hbox{$#2$}%
               \ifdim 1\wd1<1\wd2 % 
                    { \rlap{\mbox{\kern0.5\wd2\kern-0.5\wd1 \box1}} \box2 }
              \else {   \rlap{\mbox{$#1$}}
                        \rlap{\mbox{\kern0.5\wd1\kern-0.5\wd2 \box2}}
                        \phantom{#1} } 
              \fi }}

\def\rbox{\boxminus}
\def\rdia{\overstrike{\Diamond}{-}}
\def\cdia{\overstrike{\Diamond}{\rule{0.04em}{1.5ex}}}
\def\cbox{\inbox{\rule{0.04em}{1.6ex}}}
\def\cbox{\mathop{\setbox0=\hbox{$\square$}
                  \overstrike{\square}{\hbox{\rule{0.04em}{1\ht0}}}
             }}

\def\diaplus{\overstrike{\Diamond}{+}}




%% Make a modal box with a symbol centred inside
\def\inbox#1{\mathop{\setbox0=\hbox{$\square$}%
                      \setbox1=\hbox{#1}%
                      \rlap{\hbox{$\square$}}
                      \rlap{\mbox{\kern0.5\wd0\kern-0.5\wd1%
                                  \box1}}%
                      \phantom{\square}}}

\def\ibox{\mathop{\square\mkern-10mu
                   \mbox{\raise0.45ex\hbox{\tiny\em i}}
                  \mkern7mu
                 }
         }
\def\cdiamond{\mathop{\Diamond\mkern-12.7mu
                   \mbox{\raise0.4ex\hbox{\tiny\em c}}
                  \mkern7mu
                 }
         }


\def\s5box{\mathop{\square\mkern-10mu
                   \mbox{\raise0.45ex\hbox{\tiny 5}}
                  \mkern7mu
                 }
         }

\def\Fbox{\mathop{\square\mkern-11mu
                   \mbox{\raise0.45ex\hbox{\tiny \bf F}}
                  \mkern2mu
                 }
         }

\def\Box{\mathop\square}
\def\Diamond{\mathop\lozenge}

\def\bigI{\mathop{%
\hbox{%
%%
\def\thickness{1pt}%
\def\width{6pt}%
\def\height{0.6em}%
\def\gap{0.15em}%
%%
\dimen0=\thickness%
\divide\dimen0 by 2%
\dimen1=\width%
\divide\dimen1 by 2%
%%
\kern\gap%
\rlap{\vrule width\width height\thickness depth0pt}%
\rlap{\kern\dimen1 \kern-\dimen0%
\vrule width\thickness height\height depth0pt}%
\raise\height\hbox{\vrule width\width height\thickness depth0pt}%
\kern\gap}}}

\def\mconv{\mathop{% 
      \mbox{\raise0.35ex\hbox{$\scriptstyle\bigcirc$}}
                  }
          }
\def\mpconv{\circleddash}



% \def\putat#1#2#3{\setbox1=\hbox{#3}%
% \rule{0pt}{0pt}%
% \vspace*{#2}%
% \hspace*{#1}%
% \rule{0pt}{0pt}%
% \hbox to 0em{\rlap{\vtop to 0ex{\hbox to 0em{#3}}}}%
% \hspace*{-#1}\vspace*{-#2}}
% %\rule{0pt}{0pt}
% %\vspace*{-\ht1}\rule{0pt}{0pt}
% %\vspace*{-\ht1}

\def\putat#1#2#3{\rlap{\hbox{%
\kern#1% 
\vtop{\hbox{}%
\hbox{\vbox to #2{}} \hbox{#3}}}}}

% use putat within \hbox
% Objects will be located relative to baseline
% at start of the \hbox
% eg:
% \hbox{
% \putat{2in}{2in}{xxxxx}%
% \putat{2in}{2in}{00000}%
% \putat{5in}{3in}{Z}%
% \putat{2in}{2in}{0}%
% \putat{5in}{4in}{x}%
% }


%\long\def\putat#1#2#3{




\def\ri{\mathclose{\raise1ex\hbox{$\smile$}}}

\def\eqtag#1{\eqno ({\bf #1})}

\def\mc:#1{\hbox{${\mathcal{#1}}$}}
\def\mb#1{{\mathbf{#1}}}
 
\def\ifff{\qquad \hbox{if and only if} \qquad}

%%\def\d{\delta}

\def\QED{$\blacksquare$}

\def\epsrcgrant{GR/K65041}
\def\VUGgrant{GR/M56807}

\def\Boxx{\boxtimes}

%% Force a math symbol to go into scriptsize
\def\mscriptsize#1{\hbox{\scriptsize $#1$}}
\def\mlarge#1{\hbox{\large $#1$}}

\def\boxtheorem#1#2{~\\
\centerline{\fbox{
\parbox{5in}{
\centerline{\bf #1}
#2
}}}
\vspace{1ex}}

\def\proof#1#2{
\begin{quotation}
\noindent {\bf Proof of #1:}
#2
\QED
\end{quotation}
}

\def\proofpar{\\ \hspace*{1.5em}}


%%%% ENVIRONMENTS

%% specify my list parameters
%% These can be overridden by declarations within
%% a document.

\def\mytopsep{1ex}
\def\mypartopsep{0ex}
\def\myitemsep{0.3ex}
\def\myparsep{0ex}
\def\mylistparindent{2em}
\def\myleftmargin{4.5em}
\def\mylabelwidth{4.5em}
\def\mylabelsep{1em}
\def\myitemindent{0em}

\def\setmylistparams{%
             \setlength{\topsep}{\mytopsep}
             \setlength{\partopsep}{\mypartopsep}
             \setlength{\itemsep}{\myitemsep}
             \setlength{\parsep}{\myparsep}
             \setlength{\listparindent}{\mylistparindent}
             \setlength{\leftmargin}{\myleftmargin}
             \setlength{\labelwidth}{\mylabelwidth}
             \setlength{\labelsep}{\mylabelsep}
             \setlength{\itemindent}{\myitemindent}
        }

\def\clistbullet{$\bullet$}

\newcounter{tempvalue}
% A compact list making environment
\newenvironment{clist}
    { %\vspace{1ex}
      \begin{list}
            {\labelitemi}
            \setmylistparams
    }
    { \end{list}\addvspace{1ex} 
    }
% A compact list making environment
% with -- as the default item tag
\newenvironment{cdashlist}
    { %\vspace{1ex}
      \begin{list}
            {--}
            {\setlength{\topsep}{0.2ex}
             \setlength{\partopsep}{0ex}
             \setlength{\itemsep}{0.3ex}
             \setlength{\parsep}{0ex}
             \setlength{\listparindent}{2em}
            }
    }
    { \end{list}\addvspace{1ex} 
    }


% compact list with wide labels for item tags
\newenvironment{taglist}[1]
    { %\vspace{1ex}
      \begin{list}
            {$\bullet$}
            {\setlength{\topsep}{0.2ex}
             \setlength{\partopsep}{0ex}
             \setlength{\itemsep}{0.3ex}
             \setlength{\parsep}{0ex}
             \setlength{\listparindent}{2em}
             \setlength{\leftmargin}{#1}
             \setlength{\labelwidth}{#1}
             %\addtolength{\labelwidth}{-1em}
             \setlength{\labelsep}{0em}
            }
    }
    { \end{list}\addvspace{1ex} 
    }

\newcounter{cenumcount}
\newenvironment{cenum}
    { %\vspace{1ex}
      \begin{list}
            {\arabic{cenumcount}.}
            {\usecounter{cenumcount}
               \setmylistparams
            }
    }
    { \end{list}\addvspace{1ex} 
    }

\newcounter{romcount}
\newenvironment{romanenum}
          { 
            \begin{list}{\roman{romcount})~~}
                        {\usecounter{romcount}
                         \setmylistparams
                        }
          }
          { \end{list} }

\newcounter{alphcount}
\newenvironment{alphenum}
          { \begin{list}{\alph{alphcount})~~}
                        {\usecounter{alphcount}}}
          { \end{list} }


%% Labelled list environment
\newenvironment{llist}[1]{%
\newcounter{#1}
\begin{list}{({\bf #1\arabic{#1}})\hspace{1em}}{\usecounter{#1}}
}
{\end{list}}

%% Labelled list continued
%% (assumes counter already exists)
\newenvironment{llistcont}[1]{%
\setcounter{tempvalue}{\value{#1}}
\begin{list}{({\bf #1\arabic{#1}})\hspace{1em}}{%
\usecounter{#1}
\setcounter{#1}{\value{tempvalue}}
}
}
{\end{list}}

%%% compact versions of llist and llistcont
\newenvironment{cllist}[1]{%
\newcounter{#1}
\begin{list}{({\bf #1\arabic{#1}})}
             { \usecounter{#1}
               \setmylistparams
             }
}
{\end{list}}

%% Labelled list continued
%% (assumes counter already exists)
\newenvironment{cllistcont}[1]{%
\setcounter{tempvalue}{\value{#1}}
\begin{list}{({\bf #1\arabic{#1}})}{%
                 \usecounter{#1}
                 \setcounter{#1}{\value{tempvalue}}
                \setmylistparams
          }
}
{\end{list}}


\def\longdef{\long\def}

%% Define a labelled list type
%% #1 name of control sequence for the list type
%% #2 prefix for numbering
\def\llisttype#1#2{%
\newcounter{#2}
\expandafter\longdef\csname #1list\endcsname##1{%
\begin{cllistcont}{#2}
##1
\end{cllistcont}}
\expandafter\def\csname #1\endcsname##1{%
\begin{cllistcont}{#2}
\item
##1
\end{cllistcont}}
}

\def\litem#1#2{%
\begin{clist}
\item[({\bf #1})] 
#2
\end{clist}}

\newenvironment{chapabst}%
{\begin{quotation}\small\noindent }{\end{quotation}}

% Split line displayed formulae
% (not actually an environment)
\newcommand{\splitdismath}[4]{%
{\setlength{\arraycolsep}{0em}
\begin{eqnarray}
& #1 & #2 \nonumber \\
 &    & #3 
\label{#4}
\end{eqnarray}
}
}

% Displayed equation with tabbing

\newcommand{\tabmathn}[2]
{ \begin{equation}
\vbox{
\vspace{-2ex}
\begin{tabbing}
#1
\end{tabbing}
\vspace{-3.5ex}
}
\label{#2}
\end{equation}
}

\newcommand{\displaytab}[1]
{ \begin{displaymath}
\vbox{
\vspace{-2ex}
\begin{tabbing}
#1
\end{tabbing}
\vspace{-3.5ex}
}
\end{displaymath}
}

\def\myeq#1{$$\eqalign{#1}$$}


%%% Reinstate \eqalign macros removed by Lamport
\catcode`\@=11
\def\eqalign#1{\null \,\vcenter {\openup \jot \m@th 
\ialign {\strut \hfil $\displaystyle{##}$&$\displaystyle {{}##}$\hfil
 \crcr #1\crcr }}\,}

\def\eqalignno#1{\displ@y \tabskip \centering 
\halign to\displaywidth {\hfil $\@lign \displaystyle{##}$\tabskip \z@skip
 &$\@lign \displaystyle {{}##}$\hfil \tabskip \centering 
 &\llap {$\@lign ##$}\tabskip \z@skip \crcr 
 #1\crcr }}

\def\leqalignno#1{\displ@y \tabskip \centering
 \halign to\displaywidth {\hfil $\@lign \displaystyle {##}$\tabskip \z@skip
 &$\@lign \displaystyle {{}##}$\hfil \tabskip \centering
 &\kern -\displaywidth \rlap {$\@lign ##$}\tabskip \displaywidth \crcr 
 #1\crcr}}

%%% Define something like \bibliography but only records the
%%% bibfiles (does not input the bbl file)
\def\bibliographyfiles#1{%
  \if@filesw
    \immediate\write\@auxout{\string\bibdata{#1}}%
  \fi}

%%% Define a separate command to input the bbl file.
\def\bibliographytext{%
  \@input@{\jobname.bbl}}


%% Suppress printing of bibitems in thebibliography
%% This is for use with the `inlinebib' package,
%% which adds fullcitations where you give the \cite command.
%% Doesn't print note
%% Could add note by changing ##2 to ##2##3 in the \bibcite argument
%% But output not quite right.
\def\suppressendbib{%
\long\def\@bibitem##1 ##2 \note ##3 \short ##4 \end{\par\if@filesw
        {\def\protect####1{\string ####1\space}%
         \let\newblock\@empty
         \immediate\write\@auxout{\string
                \bibcite{##1}{\string\@bibcall{##1}{##2}{##4}}}}\fi
        }
\def\thebibliography##1{}
}


%% Change indent space allocated by \thebibliography
%% use before \bibliography or \bibliographytext 
\def\setbibindent#1{%
\let\oldthebibliography\thebibliography
\def\thebibliography##1{\oldthebibliography{#1}}
}



\catcode`\@=12

\long\def\skipover#1{}

\long\def\correction#1{\footnote{TO CORRECT: #1}}
\def\nocorrections{\long\def\correction##1{}}
\def\question#1{{\bf Question:} {\em #1} }
\long\def\todo#1{{\bf To Do:} {\em #1} }


\def\twiddle{$\sim$}

%% Macro for URLs
%% puts them in tt format and  breaks line after dot if needed.

\skipover{
\def\urlspace{-0.5em}
\def\url#1{{\tt \urlsub#1\end}}
%\def\urldot{. \hspace{-1em}\urlsub}
\def\urldot{. \linebreak[1]\hspace\urlspace\urlsub}
\def\urlcolon{: \linebreak[1]\hspace\urlspace\hspace\urlspace\urlsub}
\def\urlslash{/ \linebreak[1]\hspace\urlspace\urlsub}
\def\urltilde{\twiddle\urlsub}
\def\urlsub#1{\ifx#1\end \let\next=\relax
      \else \ifx#1.\let\next=\urldot
      \else \ifx#1:\let\next=\urlcolon
      \else \ifx#1/\let\next=\urlslash
      \else \ifx#1~\let\next=\urltilde
      \else#1\let\next=\urlsub\fi\fi\fi\fi\fi\next}
}

% Input a file in verbatim mode (eg a program file)
\def\verbinput#1{\expandafter\begin{verbatim}\input#1}
%NB: usage: \verbinput{filename}\end{verbatim}


\def\mb#1{{\mathbf{#1}}}
\def\mbf#1{{\mathbf{#1}}}


%%% MACROS for SLIDES


\def\heading#1{%
\centerline{\shadowbox{
\large \begin{tabular}{c} #1 \end{tabular}
           }  }
\vspace{1ex minus 1ex}
}

\def\bheading#1{%
\centerline{\shadowbox{
\large\bf \begin{tabular}{c} #1 \end{tabular}
           }  }
\vspace{1ex minus 1ex}
}

\def\printlandscape{\special{landscape}}    % Works with dvips.
\def\printportrait{\special{portrait}}

\def\leedslogo#1{\centerline{
\psfig{file=uol.eps,width=#1} %% May look wrong in xdvi
}}

\def\titleslide#1#2{
\begin{slide*}
~

\vfill
\heading{#1}

\vfill
\begin{center}
        {\large\bf Brandon Bennett} 

\vspace{3ex}
        Division of AI \\
        School of Computing \\
        University of Leeds\\ 
        Leeds LS2 9JT, England \\
        {\tt brandon@comp.leeds.ac.uk} \\
\end{center}

\vfill
\centerline{
\psfig{file=uol.eps,width=0.7in} %% May look wrong in xdvi
}
\vfill
\centerline{#2}
\vfill
\end{slide*}
}


\def\nlf{\\ \mbox{} \hfill}

%%\def\fixhyphens{\usepackage[english]{babel}}

%% FIX HYPHENATION
\lefthyphenmin=2
\righthyphenmin=3

%% special commands for spell checker
\def\sic#1{#1}
\def\sico!#1!{#1}
\def\sicname#1{#1}
\def\accept#1{}
\def\acceptname#1{}

%\newenvironment{nospell}{}{}
\def\spelloff{}
\def\spellon{}


%% Check whether a macro is defined
\def\ifundefined#1{\expandafter\ifx\csname#1\endcsname\relax}

%% Define \ifpdf
%% Used in graphic stuff to check if pdf is running
%% This is taken from ifpdf.sty
\newif\ifpdf
\ifx\pdfoutput\undefined
\else
  \ifx\pdfoutput\relax
  \else
    \ifcase\pdfoutput
    \else
      \pdftrue
    \fi
  \fi
\fi

\ifpdf
  \typeout{Running in PDF mode}
\else
  \typeout{Running in DVI mode}
\fi

%%%% CLEVER GRAPHIC INCLUSION

\def\pdfgraphictype{jpg} %% or perhaps pdf or png
\ifpdf   
  \def\graftype{\pdfgraphictype}  
\else
  \def\graftype{eps}
\fi


\typeout{* BB mymacros: Default graphic type set to: \graftype}

\ifpdf
   \def\optpdftex{pdftex}
   \def\optgraphic[#1]#2#3{\includegraphics[#1]{#3}} 
\else
   \def\pdfinfo#1{}
   \def\optpdftex{}
   \def\optgraphic[#1]#2#3{\includegraphics[#1]{#2}} 
\fi

\def\altgraphic[#1]#2#3#4{\optgraphic[#1]{#2.#3}{#2.#4}}

\def\figurepath{}
\def\graphswap[#1]#2{\optgraphic[#1]{\figurepath#2.eps}{\figurepath#2.\pdfgraphictype}}

%% \graphicifpdf includes graphic in pdf mode
%% otherwise just draws a box with the file name
%% I haven't really used this much.
\ifpdf
     \def\graphicifpdf[#1]#2{\includegraphics[#1]{#2}}
   \else
     \def\graphicifpdf[#1]#2{\fbox{\parbox{1in}{#1\\ #2}}}
\fi 

%\def\setaltgraphic#1#2{%
%   \def\altgraphic[##1]##2{\optgraphic[##1]{##2.#1}{##2.#1}}}

\def\graphic[#1]#2{\includegraphics[#1]{#2.\graftype}} 

\def\psfiggraphic{\renewcommand{\graphic}[2][]{\psfig{##1,file=##2.eps}}}

%%% Date functions

\def\dayth{\number\day\ifcase\day \or
st\or nd\or rd\or th\or th\or th\or th\or th\or th\or th\or 
th\or th\or th\or th\or th\or th\or th\or th\or th\or th\or
st\or nd\or rd\or th\or th\or th\or th\or th\or th\or th\or
st\fi}

\def\monthname{\ifcase\month \or
January\or February\or March\or April\or May\or June\or
July\or August\or September\or October\or November\or December\fi}


%%%% How to set up new font sizes
%% define a font which is Helvetica at 4.5 pt
%\newfont{\boxfont}{helvetica at 4.5 pt}

% How to rename or alter the References section
% \renewcommand\refname{References\footnote{The   reference  list  given
%     here is somewhat incomplete due to limited preparation time. It is
%     intended  to  add several  more  recent  references  to the  final
%     version of this paper, relating the discussion to current works on
%     ontology      development.}}      


\typeout{** finished loading mymacros.tex **}

%%%%%%%%%%%%%%%%%%%%%%%%%%%%%%%%%%%%%%%%%%%%%%%%%%%%%%%%%%%%%%%
%%  END END END         (of mymacros.tex)      END END END   %%
%%%%%%%%%%%%%%%%%%%%%%%%%%%%%%%%%%%%%%%%%%%%%%%%%%%%%%%%%%%%%%%
